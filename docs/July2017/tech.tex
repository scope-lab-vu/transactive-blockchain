\documentclass[]{article}


\usepackage{listings}
\usepackage{color}
\usepackage{standalone}

% Used in arch layers diagram
\usepackage{tikz}
\usetikzlibrary{chains}

% Used in sequence-diagram.
\usepackage{pgf-umlsd}

\lstset{breaklines=true, tabsize=3,keywordstyle=\color{blue},language=idl,}

%opening
\title{Technical Specifications of\\ Ethereum Network Testing Framework}
\author{}

\begin{document}

\maketitle

\begin{abstract}

Building a test network for Ethereum is a highly complex process with a number of steps involved that are tedious and error prone. We have developed a framework that allows for deployment and management of ethereum clients on a private test network. This allows for testing of not only ethereum client configurations, but also for testing of complex smart contract interactions and related software.

\end{abstract}

\section{Design Goals}

\section{System Architecture}

\subsection{Blockchain}

\subsection{RIAPS}

\subsection{Distributed File Storage}
Include description of a dist. file system that we are going to incorporate 

\section{Network Manager}

We have developed a python command line interface program for managing the test network. This 'Network Manager' (manager.py) is stateless, has built in help documentation for manual command line interactions, and each and every command and sub-command in the program has a --verbose flag associated with it. It was designed this way so as to promote easily integration into any existing network administration tool-chain(s). 

These commands are designed to allow for interaction with the entire test-network designed in <network-config.json> at once. If interactions with a specific host or client on a host is desired, then the python Fabric tool suite is provided. Run `fab -l` in the root directory of the Network Manager to be listed all Fabric methods given for interaction with individual hosts or clients.


\subsection{Bootnodes}

Bootnodes are the discovery mechanism in Ethereum. These nodes do not have a copy of the blockchain(s) on the network(s). Instead they provide a central location for peer discovery. 

\subsubsection{Create}
The following command allows you to create the bootnodes (Ethereum discovery service mechanism) specified in <network-config.json> on the test network. This command evaluates the number of hosts given, the number of bootnodes specified, and the starting port for all clients. 
\begin{lstlisting}[language=bash]
$ ./manager bootnodes create file=network-config.json
\end{lstlisting}

\subsubsection{Start}
The following command allows you to start all the bootnodes clients on the test network declared in the provided <network-config.json> file.
\begin{lstlisting}[language=bash]
$ ./manager bootnodes start file=network-config.json
\end{lstlisting}

\subsubsection{Stop}
The following command allows you to stop all the bootnodes clients on the test network declared in the provided <network-config.json> file
\begin{lstlisting}[language=bash]
$ ./manager bootnodes stop file=network-config.json
\end{lstlisting}

\subsubsection{Delete}
The following command allows you to delete all the bootnodes clients' data on the test network declared in the provided <network-config.json> file
\begin{lstlisting}[language=bash]
$ ./manager bootnodes delete file=network-config.json
\end{lstlisting}

\subsection{Miners}

Miners are the only nodes on the network allowed to do mining. These are required for the operation of the network because new blocks need to be generated so as to store transactions. Also they need to have a predictable frequency (which means adjusting difficulty of the required computations to create a new block) so as to provide timing constraints.

\subsubsection{Create}

The following command allows you to create the miners specified in <network-config.json> on the test network. This command evaluates the number of hosts given, the number of miners specified, the starting port for all clients, and the protocol to use for communication with the client once it has started.
\begin{lstlisting}[language=bash]
$ ./manager miners create file=network-config.json
\end{lstlisting}

\subsubsection{Start}

The following command allows you to start all the Blockchain clients on the test network declared in the provided <network-config.json> file.
\begin{lstlisting}[language=bash]
$ ./manager miners create file=network-config.json
\end{lstlisting}


\subsubsection{Stop}
The following command allows you to stop all the Blockchain clients on the test network declared in the provided <network-config.json> file.
\begin{lstlisting}[language=bash]
$ ./manager miners create file=network-config.json
\end{lstlisting}


\subsubsection{Delete}
The following command allows you to remove all of the Blockchain clients' data on the test network declared in the provided <network-config.json> file.
\begin{lstlisting}[language=bash]
$ ./manager miners create file=network-config.json
\end{lstlisting}


\subsection{Clients}

Clients are the nodes in the test network that have the potential to interact with the Blockchain that are NOT miners. Note: Mining is a separate process that we dedicate nodes to, so that IoT devices (Smart Meters, etc.) do not need to run intensive computation calculations.  

\subsubsection{Create}
The following command allows you to create the clients specified in <network-config.json> on the test network. This command evaluates the number of hosts given, the number of client types specified, the number of clients per type, and the starting port for all clients.
\begin{lstlisting}[language=bash]
$ ./manager clients create file=network-config.json
\end{lstlisting}

\subsubsection{Distribute}
The following command allows you to distribute the <static-nodes.json> that contains the bootnode(s) connection information to all Clients. Note: This includes all types declared within the 'clients' section of the <network-config.json> file, such as: Prosumer, Miners, etc.
\begin{lstlisting}[language=bash]
$ ./manager clients create file=network-config.json nodes=static-nodes.json
\end{lstlisting}

\subsubsection{Start}
The following command allows you to start all the clients on the test network declared in the provided <network-config.json> file.
\begin{lstlisting}[language=bash]
$ ./manager clients create file=network-config.json
\end{lstlisting}

\subsubsection{Stop}
The following command allows you to stop all the clients on the test network declared in the provided <network-config.json> file.
\begin{lstlisting}[language=bash]
$ ./manager clients create file=network-config.json
\end{lstlisting}

\subsubsection{Delete}
The following command allows you to delete all the clients' data on the test network declared in the provided <network-config.json> file.
\begin{lstlisting}[language=bash]
$ ./manager clients create file=network-config.json
\end{lstlisting}

\subsection{Blockchains}

Blockchains are the Distributed Ledger of the network infrastructure that we are setting up. In particular, we have to create a Genesis Block when creating any new test network. For simplicity sake, we also pre-allocate all clients (Consumers, Presumers, etc.) on the test network with a starting amount of currency. This allows all clients to begin translations upon creation of the network without requiring each client to mine in an attempt to gain some currency. 

\subsubsection{Make}

The following command allows you to create the <genesis-block.json> configuration file from the input <network-config.json>. This <genesis-block.json> defines the parameters required for creation of a new Blockchain.
\begin{lstlisting}[language=bash]
$ ./network-manager blockchains make file=network-config.json
\end{lstlisting}

\subsubsection{Create}
The following command allows you to create the actual blockchain defined in the <genesis-block.json> configuration file(Created with 'blockchains make' command).
\begin{lstlisting}[language=bash]
$ ./network-manager blockchains create file=network-config.json
\end{lstlisting}


\subsection{IDL for Actors}

\lstinputlisting[language=JAVA]{miner.aidl}

\lstinputlisting[language=JAVA]{dataStore.aidl}

\lstinputlisting[language=JAVA]{dso.aidl}

\lstinputlisting[language=JAVA]{prosumer.aidl}

\lstinputlisting[language=JAVA]{consumer.aidl}

\begin{figure}
	\includestandalone[width=\textwidth]{test-network-setup-seq-diag}%     without .tex extension
	% or use \input{mytikz}
	\caption{Sequence Diagram of Full Test Network Operation Cycle}
	\label{fig:seqdiag:testnetworksetup}
\end{figure}


\section{Test Network Architecture}

\subsection{Host Software}

\subsubsection{Blockchain Client}

\subsubsection{RIAPS}

\subsubsection{Distributed File Storage}

\subsection{Management Software}

command line python application leveraging Fabric for individual host control.



\end{document}

