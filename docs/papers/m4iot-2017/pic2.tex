\documentclass{standalone}

\usepackage{tikz}
\usepackage{lmodern}
\usetikzlibrary{shapes}
\usetikzlibrary{decorations.pathmorphing}
\usetikzlibrary{backgrounds}
\usetikzlibrary{calc}

%\setlength\oddsidemargin{0in}

\begin{document}
	
	% declare that we use 2 layers= background & main
	\pgfdeclarelayer{background}
	\pgfsetlayers{background,main}

	% define styles, similar to CSS 
	\newcommand*\ab{.4}
	\tikzset{
		net node/.style = {circle, minimum width=2.75*\ab cm, inner sep=2pt, outer sep=0pt, draw},
		net node2/.style = {circle, minimum width=2.75*\ab cm, inner sep=7pt, outer sep=0pt, draw, fill=white!100},
		net root node/.style = {net node, minimum width=3*\ab cm},
		dashed connect/.style = {line width=2pt,dashed, draw=blue!50!cyan!25!black},
		net connect/.style = {line width=3pt, draw=blue!50!cyan!25!black},
		snake it/.style = {line width=2pt,decorate, decoration=snake, draw=blue!50!cyan!25!black},
		texter/.style = {draw, rectangle,anchor=west},
	}

	\begin{tikzpicture}
		
		% draw nodes
		\foreach \i in {0,...,4} {
			% draw wrapper nodes
			\path (-90+\i*72:2.25) node (i\i) [net node] {Wrapper};
			% draw geth nodes
			\path (-90+\i*72:4.5) node (o\i) [net node2] {geth};
			% draw the json-rpc thick lines 
			\path [net connect] (i\i) -- (o\i);
		
			% draw eclipse around node pairs
			\draw let \p1=(o\i), \p2=(i\i), \n1={atan2(\y2-\y1,\x2-\x1)}, \n2={veclen(\y2-\y1,\x2-\x1)}
			% no clue why 4*\n1 works, just played with it + radius
			in ($ (o\i)!0.5!(i\i) $) ellipse [x radius=\n2/2+30pt, y radius=1.1cm,rotate=90-4*\n1];
		}
	
		% draw the perimeter lines, needs to be background due to snake line overdrawing into geth
		\begin{pgfonlayer}{background}
			% loop through list tupple
			\foreach \x/\nodename in {o/snake it, i/dashed connect}
				% draw connector lines
				\path [\nodename] (\x0) -- (\x1) -- (\x2) -- (\x3) -- (\x4) -- (\x0);
		\end{pgfonlayer}
	
		% define variables for 'key' dimensions
		\newcommand{\y}{-3.2}
		\newcommand{\x}{-3.5}
		\newcommand{\yscale}{0.7}
		\newcommand{\keylinestart}{2.1}
		\newcommand{\keylineend}{1.5}
		
		% define values tuple 'list'
		\newcommand{\mylist} { dashed connect/.5/Off-Blockchain, net connect/2.5/Management APIs, snake it/1.5/On-Blockchain }
	
		% loop through the 'list' 
		\foreach \nodename/\spacer/\text in \mylist {
	
			% draw 'sample' line
			\draw [\nodename] (\x-\keylineend,\y-\spacer*\yscale) -- (\x-\keylinestart,\y-\spacer*\yscale);	
			% draw 'sample' text
			\draw [texter] (\x-1.2,\y-\spacer*\yscale) node {\text};
		}

		\end{tikzpicture}
	
\end{document}