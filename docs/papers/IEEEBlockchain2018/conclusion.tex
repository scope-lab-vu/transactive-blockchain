
% Section 7
% Conclusion:
% - discussion 
% - future work?

%\color{red}

Smart and connected community applications require decentralized and scalable platforms due to the large number of participants and the lack of mutual trust between them. In this paper,  we introduced a transactive platform for resource allocation, called \Platform. We first formulated a general problem that can be used to represent a variety of resource allocation problems in smart and connected communities. Then, we described an efficient and trustworthy platform based on a hybrid approach, which combines the efficiency of traditional computing environments with the trustworthiness of blockchain-based smart contracts. Finally, we demonstrated the applicability of our platform using two case studies based on real-world data.

% Challenges in smart and connected communities: lack of trust, large number of participants, distributed system \newline
% To overcome these challenges, in this paper,
% \begin{itemize}
%     \item we introduced a transactive platform for resource allocation
%     \item we formulated a general problem that can be used to represent resource allocation problems in various smart and connected communities
%     \item we described an efficient and trustworthy platform based on a hybrid approach that combines the efficiency of traditional computing environments with the trustworthiness of blockchain based smart contracts
%     \item we evaluated and demonstrated the applicability of our platform using two case studies based on real-world data
% \end{itemize}
% \color{black}