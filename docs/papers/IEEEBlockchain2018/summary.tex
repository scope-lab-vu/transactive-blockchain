\begin{abstract}
Internet of Things and data sciences are fueling the development of innovative solutions for various applications in Smart and Connected Communities. These applications provide participants with the capability to exchange not only data but also resources, which raises the concerns of integrity, trust, and above all the need for fair and optimal solutions to the problem of resource allocation. Blockchain-based computational platforms may act as an enabling technology for these applications by providing immutability and distributed event chronology, enabling the implementation of ``higher-level contracts,'' which monitor system execution and provide guarantees on services that the rest of the system can depend upon. In this paper, we introduce the \Platform platform, which provides (a) distributed coordination and (b) decentralized market mechanisms as service. %We also describe the requirements and 
Based on case studies of energy trading and carpooling applications, we demonstrate the capabilities of \Platform.


% \textcolor{red}{the abstract has to be re-written}
% If the last decade viewed computational services as a \textit{utility}, 
% then surely this decade has transformed computation into a \textit{commodity}. Computation is now progressively integrated into physical networks in a seamless way, which enables smart and connected community applications to meet their latency requirements.  In this new  scenario, the boundaries between the network node, the sensor, and the actuator are blurring. Any node can be seen as part of a graph, with the capacity to serve as a computing or network router node (or both), and complex applications can possibly be distributed over this graph/network of nodes, so that the overall performance (e.g., amount of data processed over time) is significantly improved. However, the shared nature of resources and  the dependence on data-driven mechanisms make smart and connected community applications vulnerable. Classical, redundancy based mechanisms, such as self-checking pair and recovery blocks, are not applicable because of the dynamism and uncertainties---both endogenous (computation resources) and exogenous (environment). Blockchain based computational platforms may act as an enabling technology for these applications by providing immutability and distributed event chronology, enabling ``higher-level contracts'' to be written, which monitor system execution and provide guarantees on services that the rest of the system can depend upon. In this paper, we describe \Platform, which provides (a) distributed co-ordination and (b) decentralized market mechanisms as service. %We also describe the requirements and 
% Using case studies of energy distributed systems and carpooling, we demonstrate the capabilities of \Platform.
% %A use case of energy distribution systems is used as a casestudy. \Abhishek{title will be updated later.}
\end{abstract}