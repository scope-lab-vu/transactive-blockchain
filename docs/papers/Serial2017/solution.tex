%!TEX root = paper.tex
\section{Communication and Transactive Infrastructure}

% \section{Privacy-preserving Energy Transactions}

% This section describes \emph{Privacy-preserving Energy Transactions}
% (PETra), which is our solution for providing privacy to prosumers in a
% transactive energy IoT system, without compromising grid safety and
% security.

% \begin{figure*}[ht]
% \centering
% \begin{tikzpicture}[x=3.8cm, y=1.6cm, font=\small,
%   system/.style={draw, align=center, rounded corners=0.1cm, fill=black!5},
%   entity/.style={draw, align=center, rounded corners=0.1cm},
%   asset/.style={midway, align=center},
%   transfer/.style={->, >=stealth, shorten <=0.05cm, shorten >=0.05cm},
% ]
% \node[entity] (smartmeter) at (0, 1) {smart meter};
% \node[entity] (prosumer1) at (1, 1) {prosumer};
% \node[system] (mixing1) at (1.9, 1) {mixing\\service};
% \node[entity] (prosumer2) at (3, 1) {prosumer\\(anonymous address)};
% \node[system] (bidstorage) at (4, 0.5) {bid\\storage};
% \node[entity] (partner) at (3, 0) {other prosumer};
% \node[entity] (prosumer3) at (1.5, 0) {prosumer\\(anonymous address)};
% \node[entity] (smartmeter2) at (0, 0) {smart meter\\(anonymous address)};

% \draw[transfer] (smartmeter) -- node [asset, above] {\circled{1} energy\\production asset} (prosumer1);
% \draw[transfer] (prosumer1) -- node [asset, above] {\circled{2} energy\\production asset} (mixing1);
% \draw[transfer] (mixing1) -- node [asset, above] {\circled{3} energy\\production asset} (prosumer2);
% \draw[transfer, dashed] (prosumer2) -- node [asset, above right, xshift=-1.5em, yshift=0.5em] {\circled{4} energy production asset,\\energy ask} (bidstorage);
% \draw[transfer, dashed] (bidstorage) -- node [asset, below right] {energy\\ask} (partner);
% \draw[transfer, bend right=30] (partner) to node [asset, below] {\circled{5} financial asset,\\energy consumption asset} (prosumer3);
% \draw[transfer, bend right=50] (prosumer2) to node [asset, right, yshift=0.25em] {\circled{5} energy\\production asset} (partner);
% \draw[transfer] (prosumer3) -- node [asset, above] {\circled{6} financial asset,\\
% energy consumption asset} (smartmeter2);
% \end{tikzpicture}
% \caption{Simplified overview of the flow of assets from the perspective of a prosumer who sells energy.
% Note that to prevent de-anonymization, a prosumer should use multiple addresses and multiple rounds of mixing, which we have omitted from the figure for clarity of presentation.}
% \label{fig:sellFlow}
% %\vspace{-0.1in}
% \end{figure*}

% \subsection{Overview of the Trading Workflow}
% We now provide a semi-formal description of the energy trading
% workflow from the prosumers' perspective.  Subsequent subsections
% describe the assets, transactions, and services used for trading in
% more detail.

% \subsubsection{Energy Selling Workflow}
% Consider a prosumer who wishes to sell energy to another prosumer, as
% shown in Figure~\ref{fig:sellFlow}.  \Abhishek{We need to mention what is the
%   expected deadline by which this sequence of task should finish.}  As
% its first step, the prosumer withdraws an \emph{energy production
%   asset} from its smart meter.  An energy production asset represents
% a permission to sell a certain amount of energy, and it is used to
% enforce safety requirements.  If the prosumer has sufficient unsold
% production capacity, the smart meter creates and transfers a
% production asset to the prosumer using a \emph{smart meter
%   transaction} \circled{1}, which is recorded on the distributed
% ledger.

% At this point, the production asset can still be traced back to the
% prosumer since the ledger is public, at least within the grid.  To achieve anonymity, the prosumer uses a \emph{mixing service}, which could be implemented as a decentralized protocol, such as CryptoNote \cite{cryptonote},
% CoinShuffle~\cite{ruffing2014coinshuffle} or
% Xim~\cite{bissias2014sybil}.  The prosumer transfers the production
% asset to the mixing service using an \emph{energy and financial
%   transaction} \circled{2}, which is also recorded on the distributed
% ledger.  In turn, the mixing service transfers the production asset to
% an \emph{anonymous address} \circled{3}, which is randomly generated
% and controlled by the prosumer.\footnote{An \emph{address}
%   is an account-reference in distributed ledgers.}  Since the
% mixing service transfers assets from multiple prosumers to multiple
% anonymous addresses, and the anonymous addresses are
% generated by the prosumers, the assets cannot be traced back
% to the original prosumers after mixing.\footnote{For more details see Section \ref{Services} Services.}

% The prosumer can now engage in energy trading anonymously.  To find a
% trade partner, it can either post an \emph{energy ask} on the bid
% storage, or  search the storage for an acceptable \emph{energy
%   bid}.  To post an energy ask, the prosumer first proves to the
% storage service---without revealing its original identity---that it
% owns a production asset stored at an anonymous address.  
% \Aron{This sentence was added because we removed the relevant safety requirement.}
% Proving ownership prevents the prosumer from ``spamming'' the storage service
% with bogus asks.  The prosumer can then post the energy ask
% \circled{4}, which contains an anonymous communication
% identifier\footnote{We discuss communication anonymity later.}, a
% price, and a reference to the production asset.  If another prosumer,
% who is seeking to buy energy, finds the ask acceptable it can contact
% the seller using the communication identifier included in the ask.

% The seller and buyer can execute the trade by creating an energy and
% financial transaction together \circled{5}, and recording it on the
% ledger.  This transaction transfers the production asset from the
% seller to the buyer, and a \emph{financial asset} and an \emph{energy
%   consumption asset} from the buyer to the seller.  A financial asset
% represents a certain amount of money, while a consumption asset
% represents a permission to buy a certain amount of energy, which is
% used to enforce safety requirements similarly to production assets.

% Finally, the selling prosumer deposits the financial and consumption
% assets to its smart meter using an energy and financial transaction.
% To ensure that the prosumer remains anonymous, it transfers the assets
% to an anonymous address that is randomly generated and controlled by
% the smart meter \circled{6}.  Once the smart meter has received the
% assets, it credits the financial amount to and deducts the energy
% amount from the prosumer for billing purposes. 

% \subsubsection{Energy Buying Workflow}
% Consider a prosumer who would like to buy energy from another
% prosumer.  Since the trading workflow is very similar to the case of
% the selling prosumer, we will discuss only the differences.  In the
% first step, the prosumer tries to withdraw a financial asset and an
% energy consumption asset from its smart meter.  If the prosumer has
% the consumption capacity and good financial standing, the smart meter
% transfers the assets to the prosumer and adds the financial amount to
% the prosumer's bill.

% After transferring the assets through a mixing service, the prosumer
% can post an energy bid on the bid service.  To do so, it first
% proves the ownership of both the financial asset and the consumption
% asset to the service, and then posts the energy bid, which includes an
% anonymous communication identifier.  If a partner is found, the trade
% is executed as described above, with the prosumer playing the role of
% the buyer this time.

% Finally, the prosumer deposits the purchased energy production asset
% to the anonymous address of its smart meter, which credits the energy
% amount to the prosumer, for billing purposes.  Note that if the
% prosumer has not spent all of its financial assets, then the remainder
% may also be deposited back to the smart meter.

\subsection{Transactions}

In the previous subsection, we gave an overview of how transactions are used in the trading process to transfer various assets.
Here, we detail the format of these transactions, and the rules that they have to satisfy to be valid and recorded on the ledger.
We also introduce and detail the format of regulatory transactions, which the DSO uses to regulate the microgrid.

\subsubsection{Timing}

The ability to specify points or intervals in time is crucial.
For example, control signals specify how the microgrid load should change at certain points in time, energy trades specify when energy will be consumed or produced, etc.
To facilitate processing signals and transactions, we divide time into fixed-length intervals, and specify points or periods in time using these discrete timesteps.
The length of the time interval is determined based on the timing assumptions of the physical power system.
For example, the default length of the time interval may be 4 seconds, which corresponds to how frequently the control signal of the DSO typically changes.
\Abhishek{We need to add citation here. I will add that tomorrow.}
\Abhishek{What about the deadline within which the transactions should finish? Do we need to say anything here?}
\Aron{Ideally, we should discuss the timing constraints of the ledger (probably when we introduce it), but we would first need to make space for this discussion.}

\subsubsection{Assets}

Before we can discuss transactions, we must define the format of three assets that these transactions may transfer.
First, an \emph{energy production asset} (EPA) is defined by
\begin{compactitem}
\item \field{power}: non-negative amount of power to produced (for example, measured in watts),
\item \field{start}: first time interval in which energy is to be produced,
\item \field{end}: last time interval in which energy is to be produced.
\end{compactitem}
Second, an \emph{energy consumption asset} (ECA) is defined by the same fields; however, for this asset, the fields define energy consumption instead of production.
Finally, a \emph{financial asset} (FA) is defined by a single non-negative number \field{amount}, which can be denominated in either a fiat currency (e.g., US dollars) or a cryptocurrency.

\subsubsection{Energy and Financial Transactions}

Energy and financial transactions transfer energy and financial assets from one address to another.
Prosumers use these transactions for multiple purposes: to trade energy by exchanging assets with other prosumers, to prove to the bid storage that they have production or consumption capacity, to hide their identity by transferring assets to and from mixing services, and to deposit assets at their smart meter.
%
An energy and financial transaction contains the following fields:
\begin{compactitem}
\item \field{EPA\_in}: list of EPA inputs, each of which is defined by
\begin{itemize}[leftmargin=0.5em,nosep]
\item \field{out}: reference to an EPA output of a previous transaction,
\item \field{sig}: signature for referenced output,
\end{itemize}
\item \field{ECA\_in}: list of ECA inputs, each of which is defined by
\begin{itemize}[leftmargin=0.5em,nosep]
\item \field{out}: reference to an ECA output of a previous transaction,
\item \field{sig}: signature for referenced output,
\end{itemize}
\item \field{FA\_in}: list of FA inputs, each of which is defined by
\begin{itemize}[leftmargin=0.5em,nosep]
\item \field{out}: reference to an FA output of a previous transaction,
\item \field{sig}: signature for referenced output,
\end{itemize}
\item \field{EPA\_out}: list of EPA outputs, each of which is defined by
\begin{itemize}[leftmargin=0.5em,nosep]
\item \field{EPA}: an energy production asset,
\item \field{address}: address to which EPA is transferred,
\end{itemize}
\item \field{ECA\_out}: list of ECA outputs, each of which is defined by
\begin{itemize}[leftmargin=0.5em,nosep]
\item \field{EPA}: an energy consumption asset,
\item \field{address}: address to which ECA is transferred,
\end{itemize}
\item \field{FA\_out}: list of FA outputs, each of which is defined by
\begin{itemize}[leftmargin=0.5em,nosep]
\item \field{EPA}: a financial asset,
\item \field{address}: address to which FA is transferred,
\end{itemize}
\end{compactitem}
This transaction transfers the assets specified in the input lists to the addresses specified in the output lists. 
Note that assets may be divided or combined, as the input and output lists may differ in length.

An energy and financial transaction is valid (and can be recorded on the ledger) if the following three conditions hold.
\begin{itemize}[noitemsep,topsep=-\parskip]
\item None of the outputs referenced by the inputs have been spent by a transaction that has been recorded on the ledger.
\item All of the signatures are valid, which ensures that only the current owner can transfer an asset.
\item For each asset type (and for each timestep), the sum of inputs and outputs is equal.
For example, in the case of energy production assets, the condition is
\begin{align*}
& \forall t: \sum_{\substack{out \,\in\, \field{EPA\_out}:\\out.\field{EPA}.\field{start} \leq t \leq out.\field{EPA}.\field{end}}} out.\field{EPA}.\field{power} \nonumber \\
& = \sum_{\substack{in \,\in\, \field{EPA\_in}:\\in.\field{out}.\field{EPA}.\field{start} \leq t \leq in.\field{out}.\field{EPA}.\field{end}}} in.\field{out}.\field{EPA}.\field{power}  .
%
%& \forall t: \sum_{out \in \field{EPA\_out}} out.\field{EPA}.\field{power} \cdot 1_{\left\{out.\field{EPA}.\field{start} \leq t \leq out.\field{EPA}.\field{end}\right\}} \nonumber \\
%& = \sum_{in \in \field{EPA\_in}} in.\field{out}.\field{EPA}.\field{power} \cdot 1_{\left\{in.\field{out}.\field{EPA}.\field{start} \leq t \leq in.\field{out}.\field{EPA}.\field{end}\right\}} ,
\end{align*}
%where $1_x$ is equal to $1$ if $x$ is true, and it is $0$ otherwise.
%\begin{equation}
%\forall t: \sum_{out \in \left\{out' \middle| out' \in \text{ EPA outputs} \wedge out'.EPA.start \leq t \leq out'.EPA.end\right\}} out.EPA.Power = \sum_{in \in \left\{in' \middle| in' \in \text{ EPA inputs} \wedge in'.EPA.start \leq t \leq in'.EPA.end\right\}} in.EPA.Power .
%\end{equation}
The conditions for consumption and financial assets can be described formally in similar ways.
\end{itemize}
%If a transaction submitted to the ledger is valid, it will be permanently recorded.

\subsubsection{Smart-Meter Transactions}

Prosumers use smart-meter transactions to withdraw energy and financial assets from their own smart meters, before they engage in trading.
%
A transaction contains the following fields:
\begin{compactitem}
\item \field{EPA\_out}: list of EPA outputs (see above),
\item \field{ECA\_out}: list of ECA outputs (see above),
\item \field{FA\_out}: list of FA outputs (see above),
\item \field{id}: smart meter's identifier,
\item \field{sig}: smart meter's signature over the transaction.
\end{compactitem}
This transaction creates and transfers the assets to the prosumer's addresses, which are specified in the output lists.

The smart meter signs the transaction only if the prosumer is allowed to withdraw these assets.
More specifically, the amount of withdrawn assets can never exceed certain limits that are set by the DSO.
For example, in the case of EPA, the following condition must be satisfied for prosumer $i$:
\begin{equation}
\forall t: \sum_{tr \,\in\, \field{TR}_i} \sum_{\substack{out \,\in\, tr.\field{EPA\_out}:\\out.\field{EPA}.\field{start} \leq t \leq out.\field{EPA}.\field{end}}} out.\field{EPA}.\field{power} < \field{MAXEPA}_i ,
\end{equation}
where $\field{TR}_i$ is the set of smart-meter transactions recorded for prosumer $i$ and $\field{MAXEPA}_i$ is the withdrawal limit.
The condition for consumption assets is similar, based on a limit $\field{MAXECA}_i$.
For financial assets, the smart meter takes into account the amounts withdrawn and deposited, as well as the outside bill payments to the DSO.

\Aron{To address malfunctioning or compromised smart meters, we could also impose a limit on withdrawals.}
A transaction is valid if the following two conditions hold.
\begin{itemize}[noitemsep,topsep=-\parskip]
\item The smart meter identified in the transaction has been authorized by a regulatory transaction that was previously recorded on the ledger.
\item The smart meter's signature is valid (for public key, see regulatory transactions).
\end{itemize}

\subsubsection{Regulatory Transactions}

The DSO uses regulatory transactions for two purposes: to manage the set of authorized smart meters and to change the price policy.
%First, to change the set of smart meters that are authorized to sign transactions, the DSO authorizes or bans individual smart meters.
First, whenever a new smart meter is installed, the DSO notifies the microgrid by authorizing the device using a regulatory transaction.
Similarly, whenever a smart meter is deactivated (e.g., because service is stopped or the device is believed to be malfunctioning or compromised), the DSO notifies the microgrid by banning the device.
Second, to influence microgrid load, the DSO can set a price policy, which includes a price at which prosumer may buy energy from the DSO and a price at which they may sell energy to the DSO.

A regulatory transactions contain the following fields:
\begin{compactitem}
\item \field{authorize}: list of smart meters to be authorized, each of which is defined by
\begin{compactitem}
\item \field{id}: identifier of the smart meter,
\item \field{pubkey}: public key of the smart meter,
\end{compactitem}
\item \field{ban}: list of identifiers of smart meters to be banned, 
\item \field{priceConsumption}: price at which DSO sells energy,
\item \field{priceProduction}: price at which DSO buys energy,
\item \field{time}: timestep after which authorizations, bans, and price changes should take effect,
\item \field{sig}: DSO's signature over the transaction.
\end{compactitem}

A regulatory transaction of this type is valid if %the following two conditions hold:
%\begin{compactitem}
%\item 
\field{timestep} is not in the past and % specified in the transaction is in the future.
%\item 
the DSO's signature is valid.
%\end{compactitem}
%
The active prices for timestep $t$ are given by the last regulatory transaction recorded on the ledger whose \field{time} is less than $t$.
Similarly, regulatory transactions that are recorded on the ledger later override the authorizations and bans of earlier transactions.




