\documentclass[12pt,letterpaper]{article}

\usepackage[letterpaper,margin=1in]{geometry}

\usepackage{times}

\usepackage{enumitem}

\usepackage{hyperref}

\begin{document}

\section*{Executive Summary}

Power grids are undergoing major changes due to rapid growth in
renewable energy resources and improvements in battery technology.
While these changes enhance sustainability and efficiency, they also
create significant management challenges as the complexity of power
systems increases.  To tackle these challenges, decentralized
Internet-of-Things (IoT) solutions are emerging, which may arrange
local communities into transactive microgrids.  Within a transactive
microgrid, consumers with energy generation and storage capabilities
can trade energy with each other, thereby smoothing the load on the
main grid using local supply. However, it is hard to provide
security, safety, and privacy in a decentralized and transactive
energy system.  On the one hand, consumers' personal information must
be protected from their trade partners and the
system operator.  On the other hand, the system must be protected from
careless or malicious trading, which could destabilize the entire
grid.


\clearpage
\section*{Description}

\subsection{Introduction}

Power grids are undergoing major changes due to rapid
acceleration in renewable energy resources, such as wind and solar power \cite{5430489}. 
% \cite{EIA2014}
For example, $4,\!143$ megawatts of solar panels were installed in the
third quarter of 2016 \cite{seia}. This capacity is estimated to grow
from 4\% in 2015 to 29\% in 2040 \cite{Randal}. At the same time,
battery technology costs per kWh have dropped significantly
\cite{stock2015powerful}, reaching grid parity.
% \cite{bronski2015economics}
This massive integration of renewable energy requires detailed
information and visibility into all aspects of the network, making it
hard to manage, especially in the presence of variable distributed
energy resources \cite{7452738}. A different vision for the future of
power-grid operations is therefore emerging: {\em a decentralized
  system in which local communities are arranged in microgrids}
\cite{rahimi2012transactive}. In this vision, energy generation,
transmission, distribution, and even storage (\emph{e.g.}, electric
vehicles in a community) can be strategically used to balance load and
demand spikes.

Furthering the concept of microgrids, transactive energy models have
been proposed to support the next distribution system evolution
\cite{kok2016society,cox2013structured,melton2013gridwise}. Transactive
energy is a set of market-based constructs for dynamically balancing
the demand and supply across the electrical infrastructure
\cite{melton2013gridwise}. In this approach, customers on the same
feeder (\emph{i.e.}, those sharing a power line link) can operate in an
open market, trading and exchanging generated energy
locally. Distribution System Operators (DSOs) can be the custodians of
this market, while still meeting the net demand \cite{7462854}. For
example, the Brooklyn Microgrid, which was developed by LO3 Energy as
a pilot project, is a peer-to-peer market for locally generated
renewable energy.\footnote{\url{http://brooklynmicrogrid.com/}}

On one hand, transactive energy is a decentralized power system
controls problem \cite{7452738}, requiring strategic microgrid control
to maintain the stability of the community and the utility. On the
other hand, it is a distributed market problem where erroneous---as
well as malicious---transactions can create a gap between demand and
supply, eventually destabilizing the system. In both cases, however,
this system requires a distributed infrastructure comprising of smart
meters, feeders, smart inverters, utility substations, the utility
central offices, and the transmission system operator, which must
provide the necessary computation fabric to support the interplay
between the energy control and the fiscal market challenges.
Recently, demand-response systems have been enabled as applications of
IoT in smart grid~\cite{Haider2016166}. Transactive grid is the next step \cite{collier2017emerging}.

In general, the focus is now on creating a distributed IoT
infrastructure
that provides the necessary computation fabric to support the
interplay between energy control and fiscal market challenges, as
shown by Volttron \cite{katipamula2016volttron}, OpenFMB
\cite{gunthersmart}, and the Resilient Information Architecture
Platform for Smart Grid (RIAPS)
\cite{eisele2017riaps,Scott2017ICCPS}. For instance, the latter is a
distributed IoT operating system that provides the foundations for all
algorithms, isolates the hardware details from the algorithms, and
provides essential mechanisms for resource management, fault
tolerance, and security. Most of these efforts, however, focus on the
computation and distribution of information, and do not provide the
support required to handle the privacy challenges that arise from the
required information exchange in this decentralized transactive
system.

This paper assumes the existence of a distributed IoT infrastructure
and focuses on the following privacy challenges:
\begin{itemize}[itemsep=0.1\parskip,topsep=-0.75\parskip]
\item \textbf{Leakage of Energy Usage Patterns to Other Prosumers} 
 Since prosumers\footnote{We
  refer to customers as \emph{prosumers} to emphasize that they can
  not only consume energy, but may also produce it.} may purchase
  energy from each other in a transactive microgrid, transactions may
  inadvertently reveal the prosumers' detailed energy usage patterns
  to other prosumers within the microgrid.  Addressing this issue in a
  decentralized trading system is hard as it requires hiding the
  identities of trade partners from each other. In comparison, secure
  smart metering reveals the prosumers' energy usage patterns only to
  the operator.

\item \textbf{Inference of Future States of a Prosumer} 
  Transactions may reveal
  the future energy usage of a prosumer, which could be used to infer
  private information.  For example, a smart home may know that its
  inhabitants will go out in the evening (\emph{e.g.}, by looking at
  their calendar), and it may trade energy futures accordingly in the
  morning.  Without adequate privacy measures, these trades may reveal
  to other prosumers in the microgrid that the inhabitants will not be
  at home later.  Note that energy
  futures, whose delivery may happen several hours after when the
  transaction is made, can play an important role in predicting and
  controlling microgrid load.  In comparison, smart metering reveals
  only current (or past) usage.

\item \textbf{Personally Identifiable Information} 
  Transactions and energy
  usage data in a transactive microgrid are much richer sources of
  information than the simple usage data collected by smart
  meters.  Specifically, the information available in a
  transactive microgrid is a superset of what is available from smart
  metering, and it may be used to infer personal information, such as
  risk propensity and financial standing.
\end{itemize}
\vspace{0.5\parskip}

Before transactive energy systems can be deployed widely in practice,
we must address the privacy issues described above. However, addressing these
issues is hard since solutions must also satisfy security
and safety requirements, which often conflict with privacy goals.  For
example, to prevent a prosumer from destabilizing the grid through
careless of malicious energy trading, the system must check all of the
prosumer's transactions.  In a decentralized system, this requires
disseminating information, which could be used to infer the prosumer's
future energy consumption.

\subsection*{System Model}

A microgrid is a collection of prosumers (residential nodes) that are arranged within the same distribution feeder which supports exchange of power between them. A prosumer node typically includes a smart inverter and a smart meter, which control the flow of power into and out of the prosumer. Additionally, the microgrid contains a set of 
protection nodes that are  responsible for isolating faults on the feeder. Finally, a set of switching nodes, which are operated by the Distribution System Operator (DSO), control the connection of the microgrid to the rest of the distribution system. The DSO is responsible for regulating the net electric power into and out of the microgrid. Starting from this model, we next introduce the transactive microgrid model.

We describe a basic system model of decentralized 
transactive IoT microgrids.  We discuss the following components: a
distributed ledger for recording transactions, a bid storage service
that facilitates finding trade partners, a microgrid controller for
regulating the microgrid load, and smart meters for measuring the
prosumers' energy production and consumption.

\subsubsection*{Distributed Ledger}
This ledger permanently stores transactions that specify energy
trades, change regulatory policies for the microgrid, etc.  For
providing security and safety, it is crucial that transactions are
immutable, \emph{i.e.}, after a transaction has been recorded, it
cannot be modified or removed from the ledger.  To enhance fault
tolerance, however, the ledger should also be distributed.

Since a distributed ledger is maintained by multiple nodes, a key
requirement is reaching consensus on which transactions are valid and
stored on the ledger.  Moreover, this consensus must be reached
quickly and reliably, even in the presence of faulty or malicious
(\emph{e.g.}, compromised) ledger nodes.  This paper assumes that a
distributed ledger service is available, but makes no assumptions
about the ledger implementation, such as the particulars of the
consensus algorithm.  In practice, a distributed ledger can be
implemented using, \emph{e.g.}, \emph{blockchains} with proof-of-stake
consensus or a practical Byzantine fault tolerance
algorithm~\cite{castro1999practical}.

\subsubsection*{Bid Storage Service}
Although prosumers trade energy with each other directly (\emph{i.e.},
without a middleman), for the sake of scalability, we need a service
that enables prosumers to find trade partners.
We assume that there is a bid storage service that allows prosumers to
post and read energy \emph{bids} and \emph{asks}.\footnote{A
  \emph{bid} is an offer to buy at a certain price, while an
  \emph{ask} is an offer to sell at a certain price.}  This service
relieves prosumers from contacting a large number of potential trade
partners since they only communicate with the service to discover
trade partners.
To enhance scalability and reliability, this service can also be
implemented in a distributed manner, using multiple nodes.

\subsubsection*{Microgrid Controller (Distribution System Operator)}
We assume the existence of a controller at the DSO level that regulates the total load that the microgrid should present to the distribution system.
 The controller first predicts load in the microgrid
based on (1) bids and asks in the bid storage and (2) outstanding
energy trades in the ledger.  By combining this information with the
prediction for the rest of the grid, the controller produces a control
signal that specifies how much the microgrid load should be decreased
or increased.  Based on this signal, the controller then updates the
price policy for the microgrid to influence energy production and
consumption.  We also assume the presence of a secondary controller
that balances voltage and frequency in the microgrid.

\subsubsection*{Smart Meters}
To measure the prosumers' energy production and consumption, a smart
meter must be deployed at each prosumer.  In practice, these smart
meters must be tamper resistant to prevent prosumers from ``stealing
electricity'' by tampering with their meters.  After a smart meter has
measured the net amount of energy consumed by the prosumer in some
time interval, it can send this information to the DSO for billing
purposes.

\subsection*{Requirements}
We now discuss the security, safety, and privacy requirements that
must be satisfied by a transactive energy IoT system.

\subsubsection*{Security}
Security requirements ensure primarily that prosumers are billed
correctly, but they also provide necessary prerequisite properties for
safety.
More specifically, they require that
\begin{itemize}[noitemsep,topsep=-\parskip]
\item prosumers are billed correctly based on the energy prices set by
  the DSO, their energy trades, and their actual energy production and
  consumption measured by the smart meters,
\item prosumers or outside attackers cannot change microgrid
  regulatory policies that are set by the DSO, 
\item prosumers cannot back out of trades unilaterally, and they
  cannot tamper with other prosumers' trading or bidding,
\item financial and physical impact of compromised or faulty nodes is
  limited, and nodes can be banned by the DSO. 
\end{itemize}

\subsubsection*{Safety}
A careless or malicious prosumer may destabilize the grid by promising
to produce (or consume) a large amount of energy, but failing to actually
produce (or consume) it.  A significant difference between promised and
actual energy production (or consumption) can result in a large gap
between the aggregate production and consumption of the microgrid.
A large gap threatens the stability of not only the microgrid but also the main
power grid.  Therefore, prosumers should not be able to trade large
amounts of energy that they are unlikely to deliver.
Specifically, we require that 
\begin{itemize}[noitemsep,topsep=-\parskip]
\item the net amount of energy sold (or bought) by a prosumer is upper
  bounded (by a limit set by the DSO), where the net amount of
  energy sold is the difference between the amount of energy sold and
  bought by the prosumer, and the net amount of energy bought is
  defined analogously. 
\end{itemize}
In practice, the DSO can set the limits based on the prosumers' production and consumption capacities.

\subsubsection*{Privacy} 
Privacy requirements ensure that the prosumers' privacy is not
compromised when they participate in energy trading.  We use
non-transactive smart metering as a baseline, and we require that
the transactive system does not leak any additional information
compared to this baseline.  More specifically, we require that
\begin{itemize}[noitemsep,topsep=-\parskip]
\item only the corresponding smart meter and the DSO may gain
  information regarding the amount of energy produced, consumed,
  bought, or sold by a prosumer,\footnote{Note that this requirement
    is impossible to satisfy if all other prosumers may collude
    against one target. However, we can assume that the majority of
    prosumers are non-colluding.}
\item only the prosumer may know which bids and asks it has posted,
  and no one can know who traded energy with whom.
\end{itemize}


\section{Distributed Edge Processing for Multimodal Routing}

Urban spaces are expanding at impressive rates and an efficient transportation system is a key component to any well-functioning city. Yet, as cities expand, existing infrastructures are being stressed to their limits. To make the problem worse,  building new infrastructures is costly, time consuming, and disruptive. Consequently, a number of transit agencies and cities \cite{} are turning towards multimodal  solutions combining transporation operated by both cities and private users. However, given the large number of choices and available options this often leads to a dynamic programming problem that must be solved online considering serveral state variables, such as traffic congenstion, mode capacity and the personal preferences. 

Many commercially available Internet of Things (IoT) solutions for multimodal
transit focus on what is best for the individual from their local perspective. For example, Google Maps is able to chose between multiple options like busing, driving and biking for an individual user. But as the number of these locally optimal solution grows, too grows the misalignment between objectives of individual users and the overall system. This gap often results in a tragedy of the commons, where public goods such as roads become congested although no individual user has an incentive to use other modes of transportation. At the same time, an information bottleneck is also forming. Large scale data is being collected both by municipalities and users, but neither has the resources on their own to develop real-time analytics and controls necessary for a smart transportation system. Currently, very little has been done to provide an overarching solution that balances the needs of multiple parties including commercial companies, municipal service providers, and individuals.

\subsection{The privacy problem}
A solution to this problem requires a social computing and information sharing platform that overcomes the incentive gap between individuals and municipalities. This platform must offer mixed-mode routing suggestions and general system information to travelers and in turn provides service providers with high-fidelity information about how users are consuming different transportation resources. At the same time, this system must consider the investment required by the cities in the computing infrastructure required to solve the problem at scale. Alternatively, a novel proof-of-concept that utilizes the various edge computing resources available in the city, including the mobile devices of the commuters can be employed by municipalities to improve efficiency within their cities with little investment.

However, this precisely leads to the problem of secure and trustworthy computing.  Privacy of individuals is an important aspect of this solution; use of smart devices of individuals as both a data source and a computational resource  could have the effect of exposing the end-user to risk of a privacy breach. Seemingl innocuous data such as transit mode or route choice can lead to inferences of
private information such as real-time tracking of an individual's position~\cite{koufogiannis:2015aa}, likelihood of affairs~\cite{mueffelmann:2015aa}, forecasting trip destinations~\cite{dewri:2013aa}, etc.

\subsection{The Integrity Problem}
- the platform must be auditable
- all incentives must be tracked (new incentives will be recorded as transactions)
- The algorithms must be easy to manage and deploy. They must be validated for resource and power consumption. The incentives must track that.



\clearpage
\bibliographystyle{plain}
\bibliography{../IoT2017/references.bib}

\end{document}

