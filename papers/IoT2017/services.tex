\subsection{Services}

In this subsection, we describe the services on which transactions are built and which build on transactions.
We have already discussed distributed ledgers, which permanently store transactions.
Now, we will introduce anonymous communication service, mixing service for transaction anonymity, anonymous bid storage, and billing.\Aron{Revise list based on final organization!}

\subsubsection{Communication Anonymity}
Firstly, we must provide an anonymous communication layer, on which we can build all other services within our system.
Without this communication layer, transactions and bids could be easily de-anonymized based on the network identifiers of the sources (e.g., IP or MAC addresses).

We can employ well-known and widely used techniques for anonymous communication, such as \emph{onion routing}.
To build an onion network, the smart meters, prosumers, and other devices can act as onion routers, and the list of onion routers in a microgrid can be published on the ledger.
In practice, this service can built on 
%In our first implementation, we can use
 the free and open-source Tor software with private Directory Authorities.
In this case, anonymous communication addresses in bids and asks correspond to public-keys that identify Tor hidden services.
% ritter.vg: "run your own tor network"
% https://ritter.vg/blog-run_your_own_tor_network.html

\subsubsection{Transaction Anonymity}
We must provide prosumers with the ability to create and publish transactions anonymously.
More specifically, prosumers should be able to purchase or sell energy without revealing their identity; however, these transaction must also be verifiable and enforceable.

% discuss three approaches:
% - third-party based services: not really suitable
% - decentralized protocols: can work
``In this paper we propose CoinShuffle, a completely decentralized Bitcoin mixing protocol that allows users to utilize Bitcoin in a truly anonymous manner. ''~\cite{ruffing2014coinshuffle}
``As a solution, we propose Xim, a two-party mixing protocol that is compatible with Bitcoin and related virtual currencies.''~\cite{bissias2014sybil}
% - cryptographic anonymity (buit-in)
% Zerocoin

We may achieve this goal using multiple approaches for blockchain transaction anonymity:
\begin{itemize}
\item Mixing services (also known as tumblers) mix potentially identifiable assets on a blockchain with others, thereby preventing tracing individual assets back to their original source. 
In our case, assets to be mixed include virtual balances of fiat currencies as well as energy production and consumption.
\item Cryptographic anonymity for transactions is provided, for example, by Zerocoin~\cite{miers2013zerocoin}. Similarly to mixing service, Zerocoin can prevent tracing assets on a blockchain.
\end{itemize}
Using the above techniques, we can enable prosumers to trade energy anonymously (i.e., without revealing their true identities), but at the same time prevent them from altering their energy or financial balances without a valid transaction.

However, we must also ensure that 1) smart meters know the amount of energy purchased or sold by their prosumer and 2) prosumers cannot purchase or sell more energy than their capacity.
To satisfy both of these constraints, all energy trades must start with the prosumer withdrawing a certain amount of energy production or consumption from its smart meter:
\begin{itemize}
\item If a prosumer wishes to sell energy, it must first withdraw an energy asset from its smart meter using a blockchain transaction.
This transaction must be signed by the smart meter, which enables the smart meter to 1) keep track of the amount of energy traded by the prosumer as well as to 2) enforce safety requirements by limiting the amount of energy that can be withdrawn.
\item If a prosumer wishes to buy energy, it must first withdraw an energy consumption asset from its smart meter in a way similar to withdrawing an energy production asset.
\end{itemize}

\subsubsection{Bidding Anonymity}
Finally, we must provide prosumers with the ability to post energy bids and asks anonymously.
To this end, we create a storage for anonymous bids that is readable by all the prosumers in the microgrid.
Any prosumer may submit a bid to this storage; however, in order to do so, they must provide a zero-knowledge proof of owning the assets that are to be traded:
\begin{itemize}
\item To submit an energy sell bid, the prosumer must prove that it owns an energy production asset on the chain.
\item To submit an energy buy bid, the prosumer must prove that it own an energy consumption asset as well as financial assets on the chain.
\end{itemize}

\subsubsection{Billing}

The energy consumption balance $E_i^t$ of prosumer $i$ in timeslot~$t$ is
\begin{align*}
E_i^t = & \hphantom{+} \text{measured consumption} - \text{measured production} \\
 & + \text{EPA deposited by $i$} - \text{EPA withdrawn by $i$} .
\end{align*}
Notice that energy consumption assets are not necessary for billing, they are only used to enforce safety requirements.

The financial balance $F_i^t$ of prosumer $i$ in timeslot $t$, which is to be paid by the prosumer to the DSO, is
\begin{align*}
F_i^t = & \hphantom{+} \text{FA deposited by $i$ in $t$} - \text{FA withdrawn by $i$ in $t$} \\
 & + \begin{cases}
E_i^t \cdot \field{priceProduction} & \text{ if } E_i^t < 0 \\
- E_i^t \cdot \field{priceConsumption} & \text{ otherwise.} 
\end{cases}
\end{align*}

\TODO{discussion of microgrid control based on bids and trades? (update subsubsection title)}

