\subsection{Services}

In this subsection, we describe the various services that are implemented in our system. %on which transactions are built and which build on transactions.
We have already discussed the distributed ledger, which permanently stores valid transactions.
Now, we will introduce anonymous communication service, mixing service for transaction anonymity, anonymous bid storage, and smart-meter based billing.

\subsubsection{Communication Anonymity}
Firstly, we must provide an anonymous communication layer, on which we can build all the other services in our system.
Without this communication layer, transactions and bids could be easily de-anonymized based on their sources' network identifiers (e.g., IP or MAC addresses).

We can employ well-known and widely used techniques for anonymous communication, such as \emph{onion routing}~\cite{reed1998anonymous}.
To build an onion network, the smart meters, prosumers, and other devices can act as onion routers, and the list of onion routers in a microgrid can be published on the ledger.
In practice, this service can built on 
%In our first implementation, we can use
 the free and open-source Tor software with private Directory Authorities.
In this case, anonymous communication identifiers in bids and asks correspond to public-keys that identify Tor hidden services.
% ritter.vg: "run your own tor network"
% https://ritter.vg/blog-run_your_own_tor_network.html

\subsubsection{Transaction Anonymity}
Communication anonymity is necessary for anonymous trading, but it is not sufficient: if prosumers used their own accounts to transfer assets, trades would not be anonymous.
Fortunately, most distributed ledgers allow users to easily generate new addresses\footnote{The usage of the term \emph{address} varies between distributed ledgers, but our system could be implemented using any popular ledger, such as Bitcoin and Ethereum. 
Specifically, we use the term address to denote a possible destination for asset transfers. 
Assets transferred to an address can be used only by someone who knows the private key of the address (typically, the one who generated the address).} 
at random.
Since these addresses are generated randomly, they are anonymous in the sense that no one can tell who generated them.
However, if prosumers simply transferred assets to these addresses, they could be easily de-anonymized by tracing the assets back to the prosumers.

To prevent this, prosumers transfer assets to their anonymous addresses through a \emph{mixing service}. 
The mixing service prevents tracing the assets back to their original owners by mixing together multiple incoming transfers and multiple outgoing transfers, thereby hiding the connections between the prosumers and the anonymous addresses.
In practice, a mixing service can be implemented using multiple approaches.
The simplest one is to use a \emph{trusted third party}, called a cryptocurrency tumbler, which can receive and send assets.
However, anonymity in this case depends on the trustworthiness and reliability of the third party, who could easily de-anonymize the addresses.
A more secure approach is to used decentralized protocols, such as CoinShuffle~\cite{ruffing2014coinshuffle} or Xim~\cite{bissias2014sybil}.
These protocols enable participants to mix assets with each other, thereby eliminating the need for a trusted third party, which would constitute a single point of failure.
Some newer cryptocurrencies, such as Zerocoin~\cite{miers2013zerocoin},  provide built-in mixing services, which are often based on cryptographic principles and proofs.

\begin{comment}
We must provide prosumers with the ability to create and publish transactions anonymously.
More specifically, prosumers should be able to purchase or sell energy without revealing their identity; however, these transaction must also be verifiable and enforceable.

We may achieve this goal using multiple approaches for blockchain transaction anonymity:
\begin{itemize}
\item Mixing services (also known as tumblers) mix potentially identifiable assets on a blockchain with others, thereby preventing tracing individual assets back to their original source. 
In our case, assets to be mixed include virtual balances of fiat currencies as well as energy production and consumption.
\item Cryptographic anonymity for transactions is provided, for example, by Zerocoin~\cite{miers2013zerocoin}. Similarly to mixing service, Zerocoin can prevent tracing assets on a blockchain.
\end{itemize}
Using the above techniques, we can enable prosumers to trade energy anonymously (i.e., without revealing their true identities), but at the same time prevent them from altering their energy or financial balances without a valid transaction.

However, we must also ensure that 1) smart meters know the amount of energy purchased or sold by their prosumer and 2) prosumers cannot purchase or sell more energy than their capacity.
To satisfy both of these constraints, all energy trades must start with the prosumer withdrawing a certain amount of energy production or consumption from its smart meter:
\begin{itemize}
\item If a prosumer wishes to sell energy, it must first withdraw an energy asset from its smart meter using a blockchain transaction.
This transaction must be signed by the smart meter, which enables the smart meter to 1) keep track of the amount of energy traded by the prosumer as well as to 2) enforce safety requirements by limiting the amount of energy that can be withdrawn.
\item If a prosumer wishes to buy energy, it must first withdraw an energy consumption asset from its smart meter in a way similar to withdrawing an energy production asset.
\end{itemize}
\end{comment}

\subsubsection{Bidding Anonymity}
Finally, we must enable prosumers to anonymously post energy bids and asks on the bid storage service.
An anonymous bid (or ask) contains an ECA (or EPA), a price, and an anonymous communication identifier (e.g., Tor hidden service), which can be used to contact the bidding (or asking) prosumer.
To enforce safety requirements, the bid storage service must verify that the prosumer actually owns the asset to be traded.
To this end, the prosumer first has to prove that it controls the anonymous address where the asset is stored, which can be performed in multiple ways.
In many distributed ledgers, an address represents a public key, and controlling means knowing the corresponding private key.
In this case, the prosumer can prove that it controls an address by signing a challenge, which was freshly generated by the service, with the private key of the address.
Alternatively, the prosumer may also prove control by transferring zero amount of assets to a random address that was freshly generated by the service.

%Finally, we must provide prosumers with the ability to post energy bids and asks anonymously.
%To this end, we create a storage for anonymous bids that is readable by all the prosumers in the microgrid.
%Any prosumer may submit a bid to this storage; however, in order to do so, they must provide a zero-knowledge proof of owning the assets that are to be traded:
%\begin{itemize}
%\item To submit an energy sell bid, the prosumer must prove that it owns an energy production asset on the chain.
%\item To submit an energy buy bid, the prosumer must prove that it own an energy consumption asset as well as financial assets on the chain.
%\end{itemize}

\subsubsection{Smart-Meter Based Billing}
Once a prosumer has finished trading, it deposits all of its EPA, ECA, and FA to the smart meter, by transferring them to an anonymous address generated by the smart meter.
Later, during timeslot $t$, the smart meter measures the amount of energy actually consumed (or produced) by the prosumer using physical sensors.
Then, the meter can compute the prosumer's bill for timeslot $t$, which is to be paid to the DSO, as follows.

The energy consumption balance $E_i^t$ of prosumer $i$ is
\begin{align*}
E_i^t = & \hphantom{+} \text{measured net energy consumption during timeslot } t \\ %\text{measured consumption} - \text{measured production} \\
 & - \sum_{epa \,\in\, \{\text{EPA deposited by } i\}: ~ epa.\field{start} \,\leq\, epa.\field{end}} epa.\field{power} \\
 & + \sum_{epa \,\in\, \{\text{EPA withdrawn by } i\}: ~ epa.\field{start} \,\leq\, epa.\field{end}} epa.\field{power} .
% & + \text{EPA deposited by $i$} - \text{EPA withdrawn by $i$} .
\end{align*}
Notice that consumption assets are not used directly for billing, they are only used to enforce security and safety requirements.

The bill $B_i^t$ of prosumer $i$ for timeslot $t$, which is to be paid by the prosumer to the DSO, is
\begin{align*}
B_i^t = &  \text{FA withdrawn by $i$ during $t$} - \text{FA deposited by $i$ during $t$} \\
 & + \begin{cases}
- E_i^t \cdot \field{priceProduction} & \text{ if } E_i^t < 0 \\
 E_i^t \cdot \field{priceConsumption} & \text{ otherwise,} 
\end{cases}
\end{align*}
where {\field{priceProduction}} and {\field{priceConsumption}} are the prices set by the latest regulatory transactions for timeslot $t$.

