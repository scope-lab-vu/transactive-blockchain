%!TEX root = paper.tex
\subsection{Services}

We now describe the various services that are provided in PETra.
Earlier, we discussed the distributed ledger, which permanently
stores valid transactions.  Below, we introduce the anonymous
communication service, the mixing service for transaction anonymity,
the anonymous bid storage, and smart-meter based billing.

\subsubsection{Communication Anonymity}
The anonymous communication layer is the infrastructure upon which all
other anonymity services in PETra are built.  Without this communication
layer, transactions and bids could be easily de-anonymized based on
their sources' network identifiers (\emph{e.g.}, IP or MAC addresses).

We can employ well-known and widely used techniques for anonymous
communication, such as \emph{onion routing}~\cite{reed1998anonymous}.
To build an onion network, the smart meters, prosumers, and other
devices can act as onion routers, and the list of onion routers in a
microgrid can be published on the ledger.  In practice, this service
can be built on the free and open-source Tor software with private
Directory Authorities.  In this case, anonymous communication
identifiers in bids and asks correspond to public-keys that identify
Tor hidden services.
% ritter.vg: "run your own tor network"
% https://ritter.vg/blog-run_your_own_tor_network.html

\subsubsection{Transaction Anonymity}
Communication anonymity is necessary, but not sufficient, for
anonymous trading. In particular, if prosumers used their own accounts
to transfer assets, their trades would not be anonymous.  Fortunately,
most distributed ledgers allow users to easily generate new addresses
at random, which are anonymous in the sense that no one can tell who
generated them.  If prosumers simply transferred assets to these
addresses, however, they could be easily de-anonymized by tracing the
assets back to the prosumers.

To prevent this de-anonymization, prosumers transfer assets to their
anonymous addresses through a \emph{mixing service}.  The mixing
service prevents tracing assets back to their original owners by
mixing together multiple incoming transfers and multiple outgoing
transfers. This service thus hides the connections between the
prosumers and the anonymous addresses.

A mixing service can be implemented using multiple approaches.  The
simplest one is to use a \emph{trusted third party}, called a
cryptocurrency tumbler, which can receive and send assets. Anonymity
in this case, however, depends on the trustworthiness and reliability
of the third party, who could easily de-anonymize the addresses.  A
more secure approach is to use decentralized protocols, such as
CoinShuffle~\cite{ruffing2014coinshuffle} or
Xim~\cite{bissias2014sybil}.  These protocols enable participants to
mix assets with each other, thereby eliminating the need for a trusted
third party.  Some newer cryptocurrencies, such as
Zerocoin~\cite{miers2013zerocoin}, provide built-in mixing services,
which are often based on cryptographic principles and proofs.

\subsubsection{Bidding Anonymity}
Prosumers must also be able to anonymously post energy bids and asks
on the bid storage service.  An anonymous bid (or ask) contains an ECA
(or EPA), a price, and an anonymous communication identifier
(\emph{e.g.}, Tor hidden service), which can be used to contact the
bidding (or asking) prosumer.  To enforce safety requirements, the bid
storage service must verify that the prosumer actually owns the asset
to be traded.  To this end, the prosumer first has to prove that it
controls the anonymous address where the asset is stored, which can be
performed in multiple ways.

In many distributed ledgers, an address represents a public key, and
controlling means knowing the corresponding private key.  In this
case, the prosumer can prove that it controls an address by signing a
challenge, which was freshly generated by the service, with the
private key of the address.  Alternatively, the prosumer may also
prove control by transferring zero amount of assets to a random
address that was freshly generated by the service.

\subsubsection{Smart-Meter Based Billing}
After a prosumer has finished trading, it deposits all of its EPA,
ECA, and FA to the smart meter by transferring them to an anonymous
address generated by the smart meter.  Later, during timeslot $t$, the
smart meter measures the amount of energy actually consumed (or
produced) by the prosumer using physical sensors.  The meter can then
compute the prosumer's bill for timeslot $t$, which will be paid to
the DSO, as follows. \Abhishek{The safety argument should be
  strengthened here}

The energy consumption balance $E_i^t$ of prosumer $i$ is
\begin{align*}
E_i^t = & \hphantom{+} \text{measured net energy consumption during timeslot } t \\ 
 & - \sum_{epa \,\in\, \{\text{EPA deposited by } i\}: ~ epa.\field{start} \,\leq\, t \,\leq\, epa.\field{end}} epa.\field{power} \\
 & + \sum_{epa \,\in\, \{\text{EPA withdrawn by } i\}: ~ epa.\field{start} \,\leq\, t \,\leq\, epa.\field{end}} epa.\field{power} .
\end{align*}
Notice that consumption assets are not used directly for billing, they are only used to enforce security and safety requirements.

The bill $B_i^t$ of prosumer $i$ for timeslot $t$, which will be paid
by the prosumer to the DSO, is
\begin{align*}
B_i^t = &  \text{FA withdrawn by $i$ during $t$} - \text{FA deposited by $i$ during $t$} \\
 & + \begin{cases}
- E_i^t \cdot \field{priceProduction} & \text{ if } E_i^t < 0 \\
 E_i^t \cdot \field{priceConsumption} & \text{ otherwise,} 
\end{cases}
\end{align*}
where {\field{priceProduction}} and {\field{priceConsumption}} are the
prices set by the latest regulatory transactions for timeslot $t$. 

