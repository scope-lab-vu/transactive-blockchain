%!TEX root = paper.tex
\section{Related Work}
\label{sec:related}

%In this section we describe the related research. It has been divided into two subsections to focus on blockchains, which we are using to implement the distributed ledger service described earlier, and other works on smart grid privacy concerns. 

%\subsection{Smart Grid and Meter Privacy}
% http://ieeexplore.ieee.org/abstract/document/5054916/
%McDaniel and McLaughlin discuss privacy challenges in smart grids~\cite{mcdaniel2009security}.
% https://arxiv.org/pdf/1108.2234.pdf
% https://pdfs.semanticscholar.org/cdf8/a5b6256823bca38a1d2347ab36f8e4a2ca94.pdf
% http://www.comm.toronto.edu/~akhisti/sm.pdf
% https://www.researchgate.net/profile/Georgios_Kalogridis/publication/224189766_Smart_Grid_Privacy_via_Anonymization_of_Smart_Metering_Data/links/541169510cf2b4da1bec4193.pdf
% https://arxiv.org/pdf/1305.0735.pdf
New privacy concerns arise with the continuing adoption of smart
grids. In addition to old and new security threats (such as energy
theft and smart-meter malware), McDaniel and McLaughlin discuss the
privacy concerns of energy usage profiling that smart grids could
potentially enable~\cite{mcdaniel2009security}. Several approaches
have been investigated as potential means to provide privacy
protections for smart grid users.

Some approaches look to the use of protocols and/or frameworks to help
protect privacy. Rajagopalan et al.\ use tools from information theory
to present a framework that abstracts both the privacy and the utility
requirements of smart-meter
data~\cite{rajagopalan2011smart,sankar2013smart}. Their framework
leads to a novel tractable privacy-utility tradeoff problem with
minimal assumptions. Efthymiou and Kalogridis describe a method for
securely anonymizing frequent electrical metering data sent by a smart
meter~\cite{efthymiou2010smart}. Their approach is based on the
observation that frequent metering data may be required by an energy
distribution network for operational reasons, but it may not
necessarily need to be attributable to a specific smart meter. The
authors describe a method that provides a third-party escrow mechanism
for authenticated anonymous meter readings, which are hard to
associate with a particular smart meter.

Other approaches, such as additional hardware components, are explored
for potential privacy gains. Varodayan and Khisti study using a
rechargeable battery for partially protecting the privacy of
information contained in a household's electrical load
profile~\cite{varodayan2011smart}. They show that stochastic battery
policies may leak 26\% less information than a best-effort policy,
which holds the output load constant whenever possible. Tan et
al.\ study privacy in a smart metering system from an information
theoretic perspective in the presence of energy harvesting and storage
units~\cite{tan2013increasing}. They show that energy harvesting
provides increased privacy by diversifying the energy source, while a
storage device can be used to increase both energy efficiency and
privacy.
% They show that there exists a trade-off between the information
% leakage rate and the wasted energy rate, and study the impact of the
% energy harvesting rate and the size of the storage device on this trade-off.

PETra extends this earlier work by (1) leveraging a decentralized IoT
system for transactive energy and (2) addressing the novel privacy
threat posed by trading. In particular, while earlier work protected
the prosumers' privacy from the DSO, PETra also protects it from other
prosumers, as well as outside attackers.

A key element of PETra is its ability to distribute information among
peers via blockchains.  As blockchain technology develops and matures,
new frameworks, services, and protocols are being developed to
leverage the distributed ledgers provided by blockchains. For example,
Hyperledger Fabric is a platform for distributed ledger solutions,
which was designed to support pluggable implementations of different
components~\cite{hyperledger2017fabric}.
%Bitcoin Lightning Network is
%decentralized system, in which transactions are sent over a network of
%micropayment channels, whose transfer of value occurs
%off-blockchain~\cite{poon2016bitcoin}. 
Since this paper focuses on the theoretical foundations of PETra, any
of these distributed ledgers provide the required capabilities.

%\Abhishek{Aron this section should end with a few sentence about how our approach fits in. Perhaps just a few sentences reworded from introduction will be sufficient}

\iffalse
\subsection{Blockchains as Distributed Ledgers}

As Blockchain technology continues to develop and mature, new
frameworks, services, and protocols are being developed to leverage
blockchain's distributed ledger. Microsoft offers Blockchain as a
Service (BaaS) on Azure. Additionally, Microsoft has Project Bletchley
as its architectural approach to building an Enterprise Consortium
Blockchain Ecosystem, introducing two new concepts: blockchain
middleware and a secure means for calling code or data outside a
SmartContract or blockchain called
cryptlets~\cite{gray2016introducing}. Hyperledger Fabric is a platform
for distributed ledger solutions, which was designed to support
pluggable implementations of different
components~\cite{hyperledger2017fabric}. Bitcoin Lightning Network is
decentralized system, in which transactions are sent over a network of
micropayment channels whose transfer of value occurs
off-blockchain~\cite{poon2016bitcoin}. Interledger is a protocol for
payments across payment systems, which enables anyone with accounts on
two ledgers to create a connection between
them~\cite{thomas_protocol}. Xu et al.\ discuss the use of two pools
of proxy agents, an agreement pool and a payment pool, to assist in
protection of privacy when using blockchain technologies for
transactions on tangible goods~\cite{Xu2017}.  \fi

%\url{https://geli.net/residential/}
%\url{https://www.greentechmedia.com/articles/read/geli-raises-7m-to-take-energy-storage-software-to-the-next-level}

%\url{http://ethembedded.com/}
