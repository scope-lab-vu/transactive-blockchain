\subsection{Transactions}

\subsection{Timing}

The ability to specify points or intervals in time is crucial.
For example, control signals specify how the load should change at certain points in time, energy trades specify when energy will be consumed or produced, etc.
To facilitate processing signals and transactions, we divide time into fixed-length intervals, and specify points or periods in time using these discrete timesteps.
The length of the time interval is determined by mapping the timing assumptions of the power system to our platform.
%In our implementation, 
For example, the default length of the time interval may be 4 seconds, which corresponds to how frequently the control signal of the DSO typically changes.

\subsubsection{Assets}

Before we can discuss transaction types, we must first define the three assets that these transactions may transfer.

An \emph{energy production asset} (EPA) is defined by
\begin{compactitem}
\item \field{power}: non-negative amount of power to produced (for example, measured in watts).
\item \field{start}: first time interval in which energy is to be produced. 
\item \field{end}: last time interval in which energy is to be produced.
\end{compactitem}
An \emph{energy consumption asset} (ECA) is defined by the same fields; however, for this asset, the fields define consumption instead of production.
Finally, a \emph{financial asset} (FA) is defined by a single non-negative field \field{amount}, which can be denominated in either a fiat currency (e.g., US dollars) or a cryptocurrency.

\subsubsection{Energy and Financial Transactions}

Energy and financial transactions transfer energy and financial asset from one address to another.
Prosumers use these transactions for multiple purposes: to trade energy by exchanging assets with other prosumers, to prove to the bid storage that they have production or consumption capacity, to hide their identity by transferring assets to and from mixing services, and to deposit assets at their smart meter.
%
An energy and financial transaction contains the following fields:
\begin{compactitem}
\item List of EPA inputs, each of which is defined by:
\begin{compactitem}
\item \field{out}: EPA output from a previous transaction,
\item \field{sig}: signature for this output.
\end{compactitem}
\item List of ECA inputs, each of which is defined by:
\begin{compactitem}
\item \field{out}: ECA output from a previous transaction,
\item \field{sig}: signature for this output.
\end{compactitem}
\item List of FA inputs, each of which is defined by:
\begin{compactitem}
\item \field{out}: FA output from a previous transaction,
\item \field{sig}: signature for this output.
\end{compactitem}
\item List of EPA outputs, each of which is defined by:
\begin{compactitem}
\item an EPA and an address to which it is transferred.
\end{compactitem}
\item List of ECA outputs, each of which is defined by:
\begin{compactitem}
\item an ECA and an address to which it is transferred.
\end{compactitem}
\item List of FA outputs, each of which is defined by:
\begin{compactitem}
\item an FA and an address to which it is transferred.
\end{compactitem}
\end{compactitem}

An energy and financial transaction is valid if the following three conditions hold.
\begin{compactitem}
\item None of the outputs referenced by the inputs have been spent by a transaction that has been recorded on the ledger.
\item All of the signatures are valid.
\item For each asset type (and for each timestep), the sum of inputs and outputs is the same.
For example, in the case of energy production assets, the condition is:
\begin{align}
\forall t: & \sum_{out \in \text{EPA outputs}} out.\mathtt{power} \cdot 1_{\left\{out.\mathtt{start} \leq t \leq out.\mathtt{end}\right\}} \nonumber \\
& = \sum_{in \in \text{EPA inputs}} in.\mathtt{out}.\mathtt{power} \cdot 1_{\left\{in.\mathtt{out}.\mathtt{start} \leq t \leq in.\mathtt{out}.\mathtt{end}\right\}} ,
\end{align}
where $1_x$ is equal to $1$ if $x$ is true, and it is $0$ otherwise.
%\begin{equation}
%\forall t: \sum_{out \in \left\{out' \middle| out' \in \text{ EPA outputs} \wedge out'.EPA.start \leq t \leq out'.EPA.end\right\}} out.EPA.Power = \sum_{in \in \left\{in' \middle| in' \in \text{ EPA inputs} \wedge in'.EPA.start \leq t \leq in'.EPA.end\right\}} in.EPA.Power .
%\end{equation}
\end{compactitem}
If a transaction submitted to the ledger is valid, it will be permanently recorded.

\subsubsection{Smart-Meter Transactions}

Prosumers use smart-meter transactions to withdraw energy and financial assets from their own smart meters, before they engage in energy trading.
%
A smart-meter transaction contains the following fields:
\begin{compactitem}
\item List of EPA outputs, each of which is defined by:
\begin{compactitem}
\item an EPA and an address to which it is transferred.
\end{compactitem}
\item List of ECA outputs, each of which is defined by:
\begin{compactitem}
\item an ECA and an address to which it is transferred.
\end{compactitem}
\item List of FA outputs, each of which is defined by:
\begin{compactitem}
\item an FA and an address to which it is transferred.
\end{compactitem}
\item \field{id}: Identifier of the smart meter.
\item \field{sig}: Smart meter's signature over the transaction.
\end{compactitem}

A smart-meter transaction is valid if the following two conditions hold:
\begin{compactitem}
\item The smart meter identified in the transaction has been authorized by a regulatory transaction that has been recorded on the ledger.
\item The smart meter's signature is valid.
\end{compactitem}
\todo{To address malfunctioning or compromised smart meters, we could also impose a limit on withdrawals.}

\subsubsection{Regulatory Transactions}

The DSO uses regulatory transactions to authorize or ban smart meters and to control the microgrid load through setting a price policy.
Since these transactions provide a diverse set of functionality, they are divided into two types.

\TODO{For brevity, combine regulatory transactions into one type.}

\paragraph{Authorize or Ban Smart Meters}
The DSO uses these transactions to manage the set of authorized smart meters in the microgrid.
Whenever a new smart meter is installed, the DSO notifies the members of the microgrid using a regulatory transaction.
Similarly, whenever a smart meter is deactivated (e.g., because service is stopped or it is believed to be malfunctioning or compromised), the DSO notifies the members of the microgrid using a regulatory transaction.
These transactions contain the following fields:
\begin{compactitem}
\item List of smart meters to be authorized, each of which is defined by:
\begin{compactitem}
\item \field{id}: Identifier of the smart meter.
\item \field{pubkey}: Public key of the smart meter.
\end{compactitem}
\item List of smart meters to be banned, each of which is given using its identifier.
\item \field{time}: Timestep in which smart meters are authorized and banned.
\item \field{sig}: DSO's signature over the transaction.
\end{compactitem}

A regulatory transaction of this type is valid if the following two conditions hold:
\begin{compactitem}
\item The timestep specified in the transaction is in the future.
\item The DSO's signature is valid.
\end{compactitem}

\paragraph{Price Policy}

