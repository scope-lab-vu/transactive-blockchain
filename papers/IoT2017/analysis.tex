\section{Discussion}
\label{sec:discussion}

In this section, we present a semi-formal analysis of the proposed solution, and show that it satisfies the security, safety, and privacy requirements that were formulated earlier.

\subsection{Security}
Satisfaction of the security requirements follows from:
\begin{itemize}[noitemsep,topsep=-\parskip]
\item non-malleability of transactions in the distributed ledger,
\item validity conditions of the transactions, which include conditions on both signatures and asset balances,
\item and tamper-resistance of smart meters.
\end{itemize}
Together, these properties ensure that only the right entities may create and sign a transaction, that transactions adhere to the rules of the trading process, and that transactions cannot be tampered with.
Due to lack of space, we leave a detailed discussion and proof for future work.

\subsection{Safety}
Next, we demonstrate that faulty or malicious prosumers cannot trade excessive amounts of energy if normal prosumers follow the rules of the trading process. 
%
First, we can show that the net amount of energy sold by prosumer $i$ for each timestep is at most $\field{MAXEPA}_i$.
By definition, the amount of energy sold is less than or equal to the amount of EPA obtained by prosumer $i$.
A prosumer can obtain EPA either by withdrawing from its smart meter or by purchasing form another prosumer.
From its smart meter, prosumer $i$ can withdraw at most $\field{MAXEPA}_i$. % amount of EPA.
The prosumer may also buy EPA from another prosumer; however, this constitutes buying energy, which decreases the net amount of energy sold with the same amount.
%However, to obtain more EPA, the prosumer has to buy energy from another prosumer, which would decrease its net energy sold by the same amount.
Hence, the net amount of energy sold by prosumer $i$ cannot exceed $\field{MAXEPA}_i$.
By extending the argument, we can show that the net amount of energy sold by a group of prosumers~$G$ cannot exceed $\sum_{i \in G} \field{MAXEPA}_i$.
Similarly, we can show that the net amount of energy bought by a group of prosumers $G$ cannot exceed $\sum_{i \in G} \field{MAXECA}_i$.

Second, we can show that the amount of energy bids (or asks) posted at the same time by prosumer $i$ for each timestep is at most $\field{MAXEPA}_i + \field{MAXECA}_i$.
Notice that the limit is higher than for net energy sold or bought, since prosumer $i$ may purchase $\field{MAXECA}_i$ amount of EPA (or $\field{MAXEPA}_i$ amount of ECA) from other prosumers, and then post an energy ask (or bid) in the amount of $\field{MAXEPA}_i + \field{MAXECA}_i$.

\subsection{Privacy}

Due to the communication anonymity and mixing services, members of a microgrid can see only the amount of energy and financial assets withdrawn by a prosumer.
Since all trading transactions are anonymous, the amount of assets traded by the prosumer will not be publicly known.
In case a prosumer does not wish to trade, it can anonymously deposit its assets to a random address that was freshly generated by its smart meter.
Note that even if a prosumer does not wish to trade, it should always withdraw and mix assets; otherwise, the lack of withdrawal would leak information.

As for the DSO, it can receive the same information from the smart meter as in a non-transactive grid (i.e., amount of energy produced and consumed).
\Aron{Also knows financial amounts, but those are necessary for billing.}
Since trading is anonymous, it does not receive any further information.

In fact, we can provide an even higher-level of privacy.
Since price policies are recorded on the ledger, which the smart meters may read, the smart meters can calculate the prosumer's monthly bill, and send only this single financial amount to the DSO each month.
Meanwhile, the DSO can still obtain current and future load information from the bid storage and from the trades recorded on the ledger.

