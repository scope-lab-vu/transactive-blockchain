\section{Discussion}
\label{sec:discussion}

This section presents a semi-formal analysis of our proposed solution
and shows that it satisfies the security, safety, and privacy
requirements formulated in earlier sections\Aron{Why is ``sections'' plural? We formulated them in a single section.}.

\subsection{Security}
Satisfaction of the security requirements follows from:
\begin{itemize}[noitemsep,topsep=-\parskip]
\item non-malleability of transactions in the distributed ledger,
\item validity conditions of the transactions, which include conditions on both signatures and asset balances,
\item and tamper-resistance of smart meters.
\end{itemize}
Together, these properties ensure that only the right entities may
create and sign a transaction, that transactions adhere to the rules
of the trading process, and that transactions cannot be tampered
with.\footnote{Due to lack of space, we leave a detailed discussion
  and proof for future work.}

\subsection{Safety}
We now demonstrate that faulty or malicious prosumers cannot trade
excessive amounts of energy if normal prosumers follow the rules of
the trading process.
%
First, we can show that the net amount of energy sold by prosumer $i$ for
each timestep is at most $\field{MAXEPA}_i$.  By definition, the
amount of energy sold is less than or equal to the amount of EPA
obtained by prosumer $i$.  A prosumer can obtain EPA either by
withdrawing from its smart meter or by purchasing form another
prosumer.  From its smart meter, prosumer $i$ can withdraw at most
$\field{MAXEPA}_i$.
% amount of EPA.
Although the prosumer may also buy EPA from another prosumer, this
constitutes buying energy, which decreases the net amount of energy
sold with the same amount.
%However, to obtain more EPA, the prosumer has to buy energy from another prosumer, which would decrease its net energy sold by the same amount.
Hence, the net amount of energy sold by prosumer $i$ cannot exceed
$\field{MAXEPA}_i$.  By extending the argument, we can show that the net
amount of energy sold by a group of prosumers~$G$ cannot exceed
$\sum_{i \in G} \field{MAXEPA}_i$.  Similarly, we can show that the net
amount of energy bought by a group of prosumers $G$ cannot exceed
$\sum_{i \in G} \field{MAXECA}_i$.

We can also show that the total amount of energy bids (or asks) posted at
the same time by prosumer $i$ for each timestep is at most
$\field{MAXEPA}_i + \field{MAXECA}_i$.  This limit is higher than for
net energy sold or bought, since prosumer $i$ may purchase
$\field{MAXECA}_i$ amount of EPA (or $\field{MAXEPA}_i$ amount of ECA)
from other prosumers, and then post an energy ask (or bid) in the
amount of $\field{MAXEPA}_i + \field{MAXECA}_i$.

\subsection{Privacy}
Due to our use of communication anonymity and mixing services, members
of a microgrid can observe only the amount of assets withdrawn by a
prosumer from its smart meter.
% the amount of energy and financial assets withdrawn by a prosumer.
Since all trading transactions are anonymous, they do not reveal the
actual amount of assets traded by the prosumer.  If a prosumer does
not wish to trade, it can anonymously deposit its assets to
a random address that was freshly generated by its smart meter.  Even
if a prosumer does not wish to trade, it should always
withdraw, mix, and deposit the same amount of assets.  Otherwise, the
lack (or varying amount) of withdrawal would leak information.

As for the DSO, it receives the same information from the smart meter
as in a non-transactive smart grid (\emph{i.e.}, amount of energy
produced and consumed).  Since trading is anonymous, the DSO learns
only the financial balance of the prosumer, which is necessary for
billing.  However, we can provide an even higher-level of privacy.  In
particular, since price policies are recorded on the ledger (which the
smart meters may read), each prosumer's smart meter may calculate and
send the prosumer's monthly bill to the DSO, without revealing the
prosumer's energy consumption or production.  Meanwhile, the DSO can
still obtain detailed load information (including predictions) for the
microgrid from the bid storage and the trades recorded on the ledger.

