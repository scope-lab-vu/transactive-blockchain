\section{Preliminaries and Requirements}

\subsection{Transactive Microgrid Components}

Here, we describe a basic system model of transactive microgrids.
We discuss the following components: a distributed ledger for recording transactions, a bid storage service that facilitates finding trade partners, a microgrid controller for regulating the microgrid load, and smart meters for measuring energy production and consumption.

\subsubsection{Distributed Ledger}
The ledger permanently stores transactions that specify energy trades, change regulatory policies for the microgrid, etc.
For providing security and safety, it is crucial that transactions are non-malleable, that is, once a transaction has been recorded, it cannot be modified or removed from the ledger.
However, for the sake of fault tolerance, the ledger also needs to be distributed.
Since a distributed ledger is maintained by multiple nodes, a key implementation requirement is reaching consensus on which transactions are valid and stored in the ledger.
Further, this consensus must be reached quickly and reliably, even in the presence of faulty or malicious (e.g., compromised) ledger nodes.
In this paper, we assume that a distributed ledger service is available, but do not make any assumption about the implementation, such as the particulars of the consensus algorithm.
In practice, a distributed ledger can be implemented using, for example, \emph{blockchains} with proof-of-stake consensus or Practical Byzantine Fault Tolerance algorithm~\cite{castro1999practical}.

\subsubsection{Bid Storage Service}
Although individual prosumers trade energy with each other, for the sake of scalability, we need a service that enables prosumers to find trade partners.
%A bid storage service receives energy buy and sell bids from prosumers, stores these 
We assume that there is a bid storage service that enables prosumers to post and read energy \emph{bids} and \emph{asks}.\footnote{A \emph{bid} is an offer to buy at a certain price, while an \emph{ask} is an offer to sell at a certain price.}  
This service relieves prosumers from communicating with a large number of potential trade partners, since they only have to communicate with the service in order to discover trade partners.
Moreover, in addition to simply storing bids and asks, the service may also find matches in the posted bids and asks, and it may notify the posting prosumers of the trade opportunity.
Note that
%Even though the bid storage appears to the nodes as one entity, 
for the sake of scalability and reliability, this service can also be implemented in a distributed manner, using multiple nodes.

\subsubsection{Microgrid Controller}
The microgrid controller is responsible for stabilizing load within the microgrid and controlling it based on the expected load in the rest of the grid.
To this end, the controller first estimates the expected load in the microgrid based on the bids and asks in the bid storage as well as on outstanding energy trades in the ledger.
By combining this estimate with the expected load in the remainder of the grid, the controller produces a control signal that specifies how much the microgrid load should be decreased or increased.
Finally, based on this control signal, the controller updates the price policy for the microgrid to influence energy production and consumption.

\subsubsection{Smart Meters}
\TODO{measures energy production / consumption}
\TODO{may record energy production / consumption on the ledger or directly report it to the DSO} 

\subsection{Requirements}

\subsubsection{Security}
Prosumers (or outside attackers) should not be able to tamper with measured energy production and consumption values, with financial balances, with other prosumers' bids.
\TODO{ensure correct billing (i.e., prosumer cannot tamper with energy production / consumption or financial balances of any prosumer)}
\TODO{ensure that prosumers cannot back out of trades unilaterally, or tamper with other prosumers trading or bids}
\TODO{ensure that prosumers cannot change policies that are to be set by the DSO for the microgird}
\TODO{impact of node compromise is limited, and can be mitigated (i.e., offending node may be removed)}

\subsubsection{Safety}
A careless or malicious prosumer may destabilize the grid by promising to produce (or consume) a large amount of energy, but failing to deliver.
\TODO{this could destabilize the microgrid, or even the main grid}
Trades which are unlikely to be delivered and would result in a gap between actual production and demand should be prevented.
\TODO{net amount of energy sold (or bought) by a prosumer is limited (by a constant set by the DSO), where net amount of energy sold is the difference between the amount of energy sold and bought (net amount of energy bought is defined analogously)}
\TODO{energy bids posted by the prosumer are limited in the same way}

\subsubsection{Privacy} 
Energy trading should not compromise the privacy of prosumers.
More specifically, prosumers' private information, including energy consumption and production values, are available only to the DSO.
\TODO{other prosumers should not be able to tell how much energy was consumed / produced by a prosumer, or what bids the prosumer has posted (or with whom the prosumer traded}


