\section{Preliminaries and Requirements}

\subsection{Transactive Microgrid Components}

\Abhishek{This is the perfect paragraph to define transactive microgrid as a special application of IoT spread over a large area. Perhaps we can find a citation that makes this point.}
Here, we describe a basic system model of transactive microgrids.
We discuss the following components: a distributed ledger for recording transactions, a bid storage service that facilitates finding trade partners, a microgrid controller for regulating the microgrid load, and smart meters for measuring energy production and consumption.

\subsubsection{Distributed Ledger}
The ledger permanently stores transactions that specify energy trades, change regulatory policies for the microgrid, etc.
For providing security and safety, it is crucial that transactions are non-malleable, that is, once a transaction has been recorded, it cannot be modified or removed from the ledger.
However, for the sake of fault tolerance, the ledger also needs to be distributed.
Since a distributed ledger is maintained by multiple nodes, a key implementation requirement is reaching consensus on which transactions are valid and stored in the ledger.
Further, this consensus must be reached quickly and reliably, even in the presence of faulty or malicious (e.g., compromised) ledger nodes.
In this paper, we assume that a distributed ledger service is available, but do not make any assumption about the implementation, such as the particulars of the consensus algorithm.
In practice, a distributed ledger can be implemented using, for example, \emph{blockchains} with proof-of-stake consensus or Practical Byzantine Fault Tolerance algorithm~\cite{castro1999practical}.

\subsubsection{Bid Storage Service}
Although individual prosumers trade energy with each other, for the sake of scalability, we need a service that enables prosumers to find trade partners.
%A bid storage service receives energy buy and sell bids from prosumers, stores these 
We assume that there is a bid storage service that enables prosumers to post and read energy \emph{bids} and \emph{asks}.\footnote{A \emph{bid} is an offer to buy at a certain price, while an \emph{ask} is an offer to sell at a certain price.}  
This service relieves prosumers from communicating with a large number of potential trade partners, since they only have to communicate with the service in order to discover trade partners.
Moreover, in addition to simply storing bids and asks, the service may also find matches in the posted bids and asks, and it may notify the posting prosumers of the trade opportunity.
Note that
%Even though the bid storage appears to the nodes as one entity, 
for the sake of scalability and reliability, this service can also be implemented in a distributed manner, using multiple nodes.

\subsubsection{Microgrid Controller}
The microgrid controller is responsible for stabilizing load within the microgrid and controlling it based on the expected load in the rest of the grid.
To this end, the controller first estimates the expected load in the microgrid based on the bids and asks in the bid storage as well as on outstanding energy trades in the ledger.
By combining this estimate with the expected load in the remainder of the grid, the controller produces a control signal that specifies how much the microgrid load should be decreased or increased.
Finally, based on this control signal, the controller updates the price policy for the microgrid to influence energy production and consumption.
\Abhishek{Mention that we assume the presence of a secondary controller that balances the voltage and frequency in the microgrid.}

\subsubsection{Smart Meters}
To measure energy production and consumption, a smart meter must be deployed at each prosumer.
Further, these smart meters must be tamper resistant to prevent prosumers from ``stealing electricity'' by tampering with their meters.
Once a smart meter has measured the amount of energy produced and consumed by the prosumer, it can send this information to the DSO for billing purposes.

\subsection{Requirements}
Next, we discuss the security, safety, and privacy requirements that must be satisfied by a transactive energy IoT system.

\subsubsection{Security}
Security requirements primarily ensure that every prosumer is billed correctly, but they also provide necessary prerequisite properties for safety.
%Prosumers (or outside attackers) should not be able to tamper with measured energy production and consumption values, with financial balances, with other prosumers' bids.
More specifically, we require that 
\begin{itemize}[noitemsep,topsep=-\parskip]
\item prosumers are billed correctly based on the energy prices set by the DSO, their energy trades, and their energy production and consumption measured by the smart meters (i.e., energy production and consumption measurements and financial balances cannot be tampered with),
\item prosumers or outside attackers cannot change microgrid regulatory policies that are set by the DSO,
\item prosumers cannot back out of trades unilaterally, and they cannot tamper with other prosumers' trading and bidding,
\item financial and physical impact of a compromised or faulty node is limited, and the node can be banned by the DSO.
\end{itemize}

\subsubsection{Safety}
A careless or malicious prosumer may destabilize the grid by promising to produce (or consume) a large amount of energy, but failing to deliver.
A significant difference between promised and delivered energy production (or consumption) can result in a large gap between actual production and consumption in the microgrid, which threatens the stability of not only the microgrid but also the main power grid.
Consequently, prosumers should not be able to trade large amounts of energy, which they are unlikely to deliver.
%Trades which are unlikely to be delivered and would result in a gap between actual production and demand should be prevented.
Specifically, we require that
\begin{itemize}[noitemsep,topsep=-\parskip]
\item net amount of energy sold (or bought) by a prosumer is upper bounded (by an amount set by the DSO), where the net amount of energy sold is the difference between the amount of energy sold and bought by the prosumer, and the net amount of energy bought is defined analogously,
\item energy bids and asks posted by the prosumer are limited in a similar way.
\end{itemize}

\subsubsection{Privacy} 
Privacy requirements ensure that the prosumers' privacy is not compromised when they participate in energy trading.
We use non-transactive smart metering as a baseline, and require that the transactive system does not leak any additional information compared to the baseline.
More specifically, we require that
\begin{itemize}[noitemsep,topsep=-\parskip]
\item only the corresponding smart meter and the DSO may gain information regarding the amount of energy produced, consumed, bought, or sold by a prosumer,\footnote{Note that this requirement is impossible to satisfy if all other prosumers may collude against one target. However, we can assume that the majority of prosumers are non-colluding.}
\item only the prosumer may know which bids and asks it has posted, and no one can know who traded energy with whom.
\end{itemize}


